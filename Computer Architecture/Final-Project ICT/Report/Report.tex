\documentclass{article}
\usepackage[utf8]{inputenc}
\usepackage[export]{adjustbox}
\usepackage{geometry}
\usepackage{amsmath, amssymb}
\usepackage{listings}
\usepackage{graphicx}
\usepackage{float}
\usepackage{hyperref}
\usepackage{titlesec}
\usepackage{listings}
\usepackage{xcolor}
\usepackage{minted}

% Define background color for the code block
\definecolor{backgroundColor}{rgb}{0.95, 0.95, 0.95}

% minted settings
\setminted{
    bgcolor=backgroundColor,
    linenos,
    frame=lines,
    framesep=5pt,
    fontsize=\small
}

% Graphic Path settings
\graphicspath{{e:/Latex/image}}

% Geometry settings
\geometry{a4paper, margin=1in}

% Section formatting
\titleformat{\section}{\bfseries\large}{\thesection}{1em}{}
\titleformat{\subsection}{\bfseries\normalsize}{\thesubsection}{1em}{}

% Hyperlink settings
\hypersetup{
    colorlinks=true,
    linkcolor=blue,
    urlcolor=blue,
    pdftitle={Chap5 Notes},
    pdfauthor={},
    pdfcreator={LaTeX}
}

% Prevent automatic page breaks after minted blocks
\setlength{\parskip}{0pt}  % No paragraph skip after the block
\setlength{\parindent}{0pt} % No paragraph indent

\begin{document}
\begin{titlepage}
    \begin{center}
        \vspace*{1cm}
            
        \LARGE
        \textbf{Hanoi University of Science and Technology}
            
        \vspace{0.1cm}
        \Large
        School of Information and Communication Technology
        \rule{\linewidth}{0.4pt}
    
        \vspace{4cm}
            
        \includegraphics[width=0.3\textwidth]{university}
        
        \vspace{3cm}

        \Large
        \textbf{IT3280E - Assembly Language and Computer Architecture Lab}

        \vspace{0.25cm}
        \Large
        Final Project: Reading BMP Files and Displaying on Bitmap Screen
        \vspace{0.5cm}
        \large

        Student Name: Nguyen Duc Anh\\
        Student ID: 20235890\\
        Class: ICT-02 K68\\
        Group: 3\\
            
    \end{center}
\end{titlepage}
\section{Source Code:}
\begin{minted}{asm}
.eqv BITMAP_DISPLAY, 0x10010000   # Memmory address of Bitmap Display

.data
    # Message for user input
    input_prompt: .asciz "Enter location and .bmp image to display (Ex: D:/image/image.bmp): " 	
    # Message for resolution error handling
    error_scale: .asciz "Invalid image scale (512x512)!\n"	
    # Message for image format error handling
    error_format: .asciz "Invalid image format (.bmp)!\n"	
    # Storing user input for file path
    input_buffer: .space 256 					         
    # Memmory for debugging or fixing the code (Addition)
    debug_mem: .space 1100000					
    # Memmory to store the file content
    main_mem: .space 1100000					
    # Message for open file error handling
    error_open: .asciz "Cannot find the file!" 			
	    
.text
# Main Program
.global main
main:
    
    # Get file input
    get_input:
        li a7, 4            # Service number to print string
        la a0, input_prompt # Address of input_prompt
        ecall               # System Call

        li a7, 8            # Service number for input string
        la a0, input_buffer # Address hold the file path   
        li a1, 256          # Length of path input	
        ecall               # System Call

    # Replace "\n" with "\0"
        la t0, input_buffer         # Load the starting address of the input_buffer 
    remove_newline:		
        lb t1, (t0)                 # Load a byte from the buffer
        beq t1, zero, open_file     # Branch If the byte is '\0' -> jump to open_file
        li t2, 10                   # Load ASCII value of newline ('\n')
        beq t1, t2, replace_newline # Branch If the byte is '\n" -> jump to replace_newline
        addi t0, t0, 1              # Move to the next byte
        j remove_newline            # Repeat the process

    # Replace the character
    replace_newline:
        sb zero, (t0)               # Replace '\n' with '\0'
    
    # Open the file
    open_file:
        li a7, 1024              # Service number to open the file    
        la a0, input_buffer      # Load address of the file path 
        li a1, 0                 # Open file in Read-only mode
        ecall                    # System Call
        addi t0, a0, 0           # Store the file descriptor in t0
        blt t0, zero, file_error # Branch If file descriptor is negative -> jump to file_error 
    
    # Read the file header
    read_file:
        li   t1, 54             # BMP file header size is 54 bytes
        la   a1, main_mem       # Load address of main_mem to store the file header
        addi   a0, t0, 0        # Load file descriptor
        addi   a2, t1, 0        # Specify the length of bytes to read
        li   a7, 63             # Load syscall number for "read file"
        ecall                   # System Call
        blt  a0, t1, file_error # Branch If read fails -> jump to file_error handling
    
    # Check file format
    check_file_format:
        la   t2, main_mem         # Load address of the file header in memory
        lbu  t3, 0(t2)            # Load the first byte of the header 
        lbu  t4, 1(t2)            # Load the second byte of the header 
        li   t5, 'B'              # ASCII value for 'B'
        bne  t3, t5, format_error # Branch If first byte is not 'B' -> jump to format_error 
        li   t5, 'M'              # ASCII value for 'M'
        bne  t4, t5, format_error # Branch If second byte is not 'M' -> jump to format_error 
    
    # Check image resolution
    check_image_res:
        addi t3, t2, 18     # Move to the address of the width field in the header
        lw   t4, 0(t3)      # Load the width of the image
        addi t3, t2, 22     # Move to the address of the height field in the header
        lw   t5, 0(t3)      # Load the height of the image
    
        li t6, 512                   # Maximum scale for the image allowed is 512x512
        bgt t4, t6, resolution_error # Branch If width is invalid -> jump to resolution_error 
        bgt t5, t6, resolution_error # Branch If height is invalid -> jump to resolution_error  
    	
    # Get begin position of pixel data
    get_position:
        addi t3, t2, 10   # Move to the address of the pixel data offset in the header
        lw   t6, 0(t3)    # Load the starting position of pixel data
    
    # Move the pointer to the position of begin pixel data
    file_seek:
        addi   a0, t0, 0   # Load file descriptor	
        addi   a1, t6, 0   # Load the pixel data offset
        li   a2, 0         # Specify SEEK_SET (absolute positioning) 
        li   a7, 62        # Service number for seek
        ecall              # System Call
        blt  a0, zero, end # Branch If seek fails -> jump to end		
    
    # Calculate pixel data for display on Bitmap Display
    pixel_calculate:
        li    s10, 3       # Each pixel is 3 bytes (RGB format)
        mul   s7, t4, s10  # Calculate RowSize (Not include padding)     
        mul   s8, s7, t5   # Calculate PixelData = RowSize x Height
    
    # Read the pixel data
    read_pixel:
        la   a1, main_mem   # Load address of main_mem to store pixel data
        addi   a0, t0, 0    # Load file descriptor
        addi   a2, s8, 0    # Specify the total size of pixel data
        li   a7, 63         # Service number for read file
        ecall               # System Call       
        blt  a0, s8, end    # Branch If read fails -> jump to end
    
    # Initial Bitmap Display for output
    init:
        li a3, BITMAP_DISPLAY  # Load the Bitmap Display base address
        addi s1, t5, 0         # Height of the image
        addi s2, t4, 0         # Width of the image
    
#-----------------------------------------------------------------------------------------------
# Algorithm for displaying:                                                                    |
#                                                                                              |
# s2 = width of the image.                                                                     |
# s1 = height of the image.                                                                    |
#                                                                                              |
# Loop from bottom up for each rows.                                                           |
# This purpose is to access first row of bitmap image.                                         |
#                                                                                              | 
# For each rows, loop from left to right to access each columns refers for each pixels.        |
# For each columns (aka pixels) in the row, process the color and display immediately to the   | 
# display.                                                                                     |
#                                                                                              |
# The default color format of bitmap image it's BGR.                                           |
# Implementation of pixel in a row of bitmap image: [B1 G1 R1] [B2 G2 R2] [B3 G3 R3] ...       |
# To display image on Bitmap Display, we need to convert BGR color format into RGB color format|
# It means convert 0x00BBGGRR -> 0x00RRGGBB.                                                   |
# Each value on the hexadecimal value it's 4 bit.                                              |
#                                                                                              |
# To do that, we use shift left logic and bitwise operation to evaluate the exactly value of   |
# the color refers to RGB format to display the image with right color as accurately as        |                                                                            |
# possible.                                                                                    |
#                                                                                              |
#-----------------------------------------------------------------------------------------------
    
    # Algorithm implement
    display:    
        loop_rows:
            addi s1, s1, -1 # Decrease the row counter               
            blt  s1, zero, end # Branch If all rows of the image processed -> jump to end      
            mul s9, s1, s7 # Calculate the offset to the current row's pixel data     
            la  t3, main_mem # Get the address of the image header to access each pixel    
            add t3, t3, s9 # Get the begin address of the pixel data of the current row      
            addi s4, s2, 0 # Reset the width of the row after access each columns (aka pixels)                     
            addi s5, t3, 0 # Init pointer to process each pixels in the row           
        
        loop_cols:
            beq s4, zero, next_row # Branch if all pixels in a row processed -> jump to next row            
            lbu t1, 0(s5)     # Get the B value 0xBB     
            lbu t2, 1(s5)     # Get the G value 0xGG   
            lbu s11, 2(s5)    # Get the R value 0xRR  
            slli s11, s11, 16 # Shift left logical s10 = 0x000000RR -> 0x00RR0000
            slli t2, t2, 8    # Shift left logical t2 = 0x000000GG -> 0x0000GG00
            or   s11, s11, t2 # Logical OR: s10 or t2 -> 0x00RRGG00
            or   s11, s11, t1 # Logical OR: s10 or t1 -> 0x00RRGGBB
            sw  s11, 0(a3) # Store the value into Bitmap Display to display
            addi a3, a3, 4 # Move to the next pixel on Bitmap Display (each 4 bytes)
            addi s5, s5, 3 # Move to the next pixel on image after process 3 byte RGB format 
            addi s4, s4, -1 # Decrease the pixel counter of the image 
            j loop_cols     # Jump to next pixel
        
        next_row:
            j loop_rows     # Jump to next row

end:
    # Close the file
    close_file:
        addi a0, t0, 0  # Load the file descriptor
        li a7, 57       # Service number for clode the file
        ecall           # System Call
        
    # Exit the Program
    exit:
        li a0, 0        # Load exit code (0)
        li a7, 10       # Service number for exit program
        ecall           # System Call

# Error Handling
# Open file error handle
file_error:
    li a7, 4           # Service number for print string
    la a0, error_open  # Load address of error_open Message
    ecall              # System Call
    j end              # Jump to end

# Resolution of image error handle
resolution_error:
    li a7, 4            # Service number for print string
    la a0, error_scale  # Load address of error_scale Message 
    ecall               # System Call
    j end               # Jump to end

# Format file error handle
format_error:
    li a7, 4            # Service number for print string
    la a0, error_format # Load address of error_format Message
    ecall               # System Call
    j end               # Jump to end

\end{minted}

\section{Explanation:}
\subsection{Declare Variables}
\subsubsection{Memmory address of Bitmap Display}
\begin{itemize}
    \item \mintinline{asm}{.eqv BITMAP_DISPLAY, 0x10010000}: Defines the memory-mapped address of the bitmap display.
\end{itemize}
\subsubsection{Input Variables}
\begin{itemize}
    \item \mintinline{asm}{input_prompt: .asciz "Enter location and .bmp image to display (Ex: D:/image/image.bmp): "}: String to prompt the user to input a BMP file path.
    \item \mintinline{asm}{input_buffer: .space 256}: Allocates 256 bytes for storing user input.
\end{itemize}
\subsubsection{Error Message}
\begin{minted}{asm}
    error_scale: .asciz "Invalid image scale (512x512)!\n"
    error_format: .asciz "Invalid image format (.bmp)!\n"
    error_open: .asciz "Cannot find the file!"
\end{minted}    
\begin{itemize}
    \item Strings for error messages to handle different error conditions (resolution, format, file opening).
\end{itemize}
\subsubsection{Memmory Allocates}
\begin{itemize}
    \item \mintinline{asm}{main_mem: .space 1100000}: Allocates a large memmory region to storing the BMP file content.
    \item \mintinline{asm}{debug_mem: .space 1100000}: Allocates a large memmory region to debugging. In addition, the purpose of this is to prevent overwriting other data, align memmory and simulate reserve space.
\end{itemize}
\subsection{Main Program}
\begin{minted}{asm}
    .global main
    main:
\end{minted}
\begin{itemize}
    \item Defines the main entry point for the program.
\end{itemize}
\subsubsection{Input Handling}
\begin{minted}{asm}
    # Get file input
    get_input:
        li a7, 4            # Service number to print string
        la a0, input_prompt # Address of input_prompt
        ecall               # System Call
\end{minted}
\begin{itemize}
    \item Displays the input prompt using a syscall for printing a string.
\end{itemize}
\begin{minted}{asm}
    li a7, 8            # Service number for input string
    la a0, input_buffer # Address hold the file path   
    li a1, 256          # Length of path input	
    ecall               # System Call
\end{minted}
\begin{itemize}
    \item Accepts the user's file path input and stores it in \mintinline{asm}{input_buffer}. 
\end{itemize}
\subsubsection{Replace Newline with Null Terminator}
\begin{minted}{asm}
    # Replace "\n" with "\0"
        la t0, input_buffer         # Load the starting address of the input_buffer 
    remove_newline:		
        lb t1, (t0)                 # Load a byte from the buffer
        beq t1, zero, open_file     # Branch If the byte is '\0' -> jump to open_file
        li t2, 10                   # Load ASCII value of newline ('\n')
        beq t1, t2, replace_newline # Branch If the byte is '\n" -> jump to replace_newline
        addi t0, t0, 1              # Move to the next byte
        j remove_newline            # Repeat the process

    # Replace the character
    replace_newline:
        sb zero, (t0)               # Replace '\n' with '\0'
\end{minted}
\begin{itemize}
    \item Iterates through \mintinline{asm}{input_buffer} to find and replace the newline character (\mintinline{asm}{\n}, ASCII 10) with a null terminator \mintinline{asm}{\0}.
\end{itemize}
\subsubsection{Open File}
\begin{minted}{asm}
    # Open the file
    open_file:
        li a7, 1024              # Service number to open the file    
        la a0, input_buffer      # Load address of the file path 
        li a1, 0                 # Open file in Read-only mode
        ecall                    # System Call
        addi t0, a0, 0           # Store the file descriptor in t0
        blt t0, zero, file_error # Branch If file descriptor is negative -> jump to file_error
\end{minted}
\begin{itemize}
    \item Opens the BMP file in read-only mode. If the file descriptor is negative, it jumps to \mintinline{asm}{file_error}.
\end{itemize}
\subsubsection{Read File Header}
\begin{minted}{asm}
    # Read the file header
    read_file:
        li   t1, 54             # BMP file header size is 54 bytes
        la   a1, main_mem       # Load address of main_mem to store the file header
        addi   a0, t0, 0        # Load file descriptor
        addi   a2, t1, 0        # Specify the length of bytes to read
        li   a7, 63             # Load syscall number for "read file"
        ecall                   # System Call
        blt  a0, t1, file_error # Branch If read fails -> jump to file_error handling
\end{minted}
\begin{itemize}
    \item Reads the first 54 bytes (header) of the BMP file into \mintinline{asm}{main_mem}.
\end{itemize}
\subsubsection{Validate File Format}
\begin{minted}{asm}
    # Check file format
    check_file_format:
        la   t2, main_mem         # Load address of the file header in memory
        lbu  t3, 0(t2)            # Load the first byte of the header 
        lbu  t4, 1(t2)            # Load the second byte of the header 
        li   t5, 'B'              # ASCII value for 'B'
        bne  t3, t5, format_error # Branch If first byte is not 'B' -> jump to format_error 
        li   t5, 'M'              # ASCII value for 'M'
        bne  t4, t5, format_error # Branch If second byte is not 'M' -> jump to format_error 
\end{minted}
\begin{itemize}
    \item Checks the first two bytes of the BMP header to ensure they are 'B' and 'M'.
\end{itemize}
\subsubsection{Validate Image Resolution}
\begin{minted}{asm}
    # Check image resolution
    check_image_res:
        addi t3, t2, 18     # Move to the address of the width field in the header
        lw   t4, 0(t3)      # Load the width of the image
        addi t3, t2, 22     # Move to the address of the height field in the header
        lw   t5, 0(t3)      # Load the height of the image
    
        li t6, 512                   # Maximum scale for the image allowed is 512x512
        bgt t4, t6, resolution_error # Branch If width is invalid -> jump to resolution_error 
        bgt t5, t6, resolution_error # Branch If height is invalid -> jump to resolution_error
\end{minted}
\begin{itemize}
    \item Validates the width and height of the image (both must be 512).
\end{itemize}
\subsubsection{Read Pixel Data}
\begin{minted}{asm}
    # Get begin position of pixel data
    get_position:
        addi t3, t2, 10   # Move to the address of the pixel data offset in the header
        lw   t6, 0(t3)    # Load the starting position of pixel data
\end{minted}
\begin{itemize}
    \item Extracts the offset to the pixel data from the header.
\end{itemize}
\begin{minted}{asm}
    # Move the pointer to the position of begin pixel data
    file_seek:
        addi   a0, t0, 0   # Load file descriptor	
        addi   a1, t6, 0   # Load the pixel data offset
        li   a2, 0         # Specify SEEK_SET (absolute positioning) 
        li   a7, 62        # Service number for seek
        ecall              # System Call
        blt  a0, zero, end # Branch If seek fails -> jump to end
\end{minted}
\begin{itemize}
    \item Moves the file pointer to the pixel data.
\end{itemize}
\begin{minted}{asm}
    # Calculate pixel data for display on Bitmap Display
    pixel_calculate:
        li    s10, 3       # Each pixel is 3 bytes (RGB format)
        mul   s7, t4, s10  # Calculate RowSize (Not include padding)     
        mul   s8, s7, t5   # Calculate PixelData = RowSize x Height
\end{minted}
\begin{itemize}
    \item Calculates the total size of the pixel data.
\end{itemize}
\begin{minted}{asm}
    # Read the pixel data
    read_pixel:
        la   a1, main_mem   # Load address of main_mem to store pixel data
        addi   a0, t0, 0    # Load file descriptor
        addi   a2, s8, 0    # Specify the total size of pixel data
        li   a7, 63         # Service number for read file
        ecall               # System Call       
        blt  a0, s8, end    # Branch If read fails -> jump to end
\end{minted}
\begin{itemize}
    \item Reads the pixel data into \mintinline{asm}{main_mem}.
\end{itemize}
\subsubsection{Display Pixel Data}
\begin{minted}{asm}
    # Initial Bitmap Display for output
    init:
        li a3, BITMAP_DISPLAY  # Load the Bitmap Display base address
        addi s1, t5, 0         # Height of the image
        addi s2, t4, 0         # Width of the image
\end{minted}
\begin{itemize}
    \item Initializes the base address for the bitmap display and stores image dimensions.
\end{itemize}
\begin{minted}{asm}
#-----------------------------------------------------------------------------------------------
# Algorithm for displaying:                                                                    |
#                                                                                              |
# s2 = width of the image.                                                                     |
# s1 = height of the image.                                                                    |
#                                                                                              |
# Loop from bottom up for each rows.                                                           |
# This purpose is to access first row of bitmap image.                                         |
#                                                                                              | 
# For each rows, loop from left to right to access each columns refers for each pixels.        |
# For each columns (aka pixels) in the row, process the color and display immediately to the   | 
# display.                                                                                     |
#                                                                                              |
# The default color format of bitmap image it's BGR.                                           |
# Implementation of pixel in a row of bitmap image: [B1 G1 R1] [B2 G2 R2] [B3 G3 R3] ...       |
# To display image on Bitmap Display, we need to convert BGR color format into RGB color format|
# It means convert 0x00BBGGRR -> 0x00RRGGBB.                                                   |
# Each value on the hexadecimal value it's 4 bit.                                              |
#                                                                                              |
# To do that, we use shift left logic and bitwise operation to evaluate the exactly value of   |
# the color refers to RGB format to display the image with right color as accurately as        |                                                                            |
# possible.                                                                                    |
#                                                                                              |
#-----------------------------------------------------------------------------------------------
\end{minted}
\begin{minted}{asm}
    display:    
        loop_rows:
            addi s1, s1, -1 # Decrease the row counter               
            blt  s1, zero, end # Branch If all rows of the image processed -> jump to end      
            mul s9, s1, s7 # Calculate the offset to the current row's pixel data     
            la  t3, main_mem # Get the address of the image header to access each pixel    
            add t3, t3, s9 # Get the begin address of the pixel data of the current row      
            addi s4, s2, 0 # Reset the width of the row after access each columns (aka pixels)                     
            addi s5, t3, 0 # Init pointer to process each pixels in the row
\end{minted}
\begin{itemize}
    \item Iterates through the rows of the image, starting from the bottom.
\end{itemize}
\begin{minted}{asm}
    loop_cols:
            beq s4, zero, next_row # Branch if all pixels in a row processed -> jump to next row            
            lbu t1, 0(s5)     # Get the B value 0xBB     
            lbu t2, 1(s5)     # Get the G value 0xGG   
            lbu s11, 2(s5)    # Get the R value 0xRR  
            slli s11, s11, 16 # Shift left logical s10 = 0x000000RR -> 0x00RR0000
            slli t2, t2, 8    # Shift left logical t2 = 0x000000GG -> 0x0000GG00
            or   s11, s11, t2 # Logical OR: s10 or t2 -> 0x00RRGG00
            or   s11, s11, t1 # Logical OR: s10 or t1 -> 0x00RRGGBB
            sw  s11, 0(a3) # Store the value into Bitmap Display to display
            addi a3, a3, 4 # Move to the next pixel on Bitmap Display (each 4 bytes)
            addi s5, s5, 3 # Move to the next pixel on image after process 3 byte RGB format 
            addi s4, s4, -1 # Decrease the pixel counter of the image 
            j loop_cols     # Jump to next pixel
\end{minted}
\begin{itemize}
    \item Converts the pixel format from BGR to RGB and writes it to the display.
\end{itemize}
\begin{minted}{asm}
    next_row:
            j loop_rows     # Jump to next row
\end{minted}
\begin{itemize}
    \item Moves to the next row of the image.
\end{itemize}
\subsubsection{End Program}
\begin{minted}{asm}
end:
    # Close the file
    close_file:
        addi a0, t0, 0  # Load the file descriptor
        li a7, 57       # Service number for clode the file
        ecall           # System Call   
\end{minted}
\begin{itemize}
    \item Execute close the file after processed.
\end{itemize}
\begin{minted}{asm}
    # Exit the Program
    exit:
        li a0, 0        # Load exit code (0)
        li a7, 10       # Service number for exit program
        ecall           # System Call
\end{minted}
\begin{itemize}
    \item Exit the program after finished all.
\end{itemize}
\subsection{Error Handling}
\begin{minted}{asm}
# Error Handling
# Open file error handle
file_error:
    li a7, 4           # Service number for print string
    la a0, error_open  # Load address of error_open Message
    ecall              # System Call
    j end              # Jump to end

# Resolution of image error handle
resolution_error:
    li a7, 4            # Service number for print string
    la a0, error_scale  # Load address of error_scale Message 
    ecall               # System Call
    j end               # Jump to end

# Format file error handle
format_error:
    li a7, 4            # Service number for print string
    la a0, error_format # Load address of error_format Message
    ecall               # System Call
    j end               # Jump to end
\end{minted} 
\begin{itemize}
    \item Prints the corresponding error message and jump to end process.
\end{itemize}
\section{Result Observe:}
\includegraphics[width=1\textwidth]{Final}
\end{document}